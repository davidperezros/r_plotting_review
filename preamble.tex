

% Para el titulo
\usepackage{titling}
\pretitle{\begin{center}
\href{https://www.ine.es}{\includegraphics[width=2in,height=2in]{fig/ine_logo.pdf}}\LARGE\\}
\posttitle{\end{center}}
% Para poder definir el background color de los chunks de R
% al generar el pdf
\usepackage{xcolor}
\definecolor{granate_ine}{RGB}{136,19,51}
\definecolor{verde_fuerte_ine}{RGB}{69,126,118}
\definecolor{verde_claro_ine}{RGB}{221,238,236}

\usepackage{etoolbox}
\patchcmd{\tableofcontents}{Contents}{\'Indice}{}{}

% Para los márgenes del pdf
\usepackage[a4paper,
            left=1in,
            right=1in,
            top=1.2in,
            bottom=1in,
            footskip=.25in]{geometry}
% Para cambiar el color de los títulos a granate INE
\usepackage{sectsty}
\chapterfont{\color{granate_ine}}  % sets colour of chapters
\sectionfont{\color{granate_ine}}  % sets colour of sections
\subsectionfont{\color{granate_ine}}  % sets colour of subsections


% Ponemos en español estas palabras
\renewcommand{\contentsname}{Indice}
\renewcommand{\chaptername}{Capítulo}

% Ponemos todo el texto en negrita como el verde INE
\let\oldtextbf\textbf
\renewcommand{\textbf}[1]{\textcolor{verde_fuerte_ine}{\oldtextbf{#1}}}




% Para las cajas de información en pdf

\usepackage{tcolorbox}

\newtcolorbox{blackbox}{
  colback=verde_claro_ine,
  colframe=granate_ine,
  coltext=verde_fuerte_ine,
  boxsep=5pt,
  arc=4pt}

\newenvironment{infobox}[1]
  {
  \begin{itemize}
  \renewcommand{\labelitemi}{
    \raisebox{-.7\height}[0pt][0pt]{
      {\setkeys{Gin}{width=3em,keepaspectratio}
        \includegraphics{fig/caution.png}}
    }
  }
  \setlength{\fboxsep}{1em}
  \begin{blackbox}
  \item
  }
  {
  \end{blackbox}
  \end{itemize}
  }
